\documentclass{article}
\usepackage{mystyle}

\title{Amplicon Based Somatic Variant Calling}
\author{Edmund Lau}
\date{}


\begin{document}
\maketitle
\section{Background}
We are shall call point base substitution type mutation only. 


\section{Per locus error estimate}
Let 
\begin{align*}
	\A = \set{A, T, G, C}
\end{align*}
denote the set of possible alleles\footnote{The preferred ordering is $(A, T, G, C)$} at any particular genomic locus of interest 
\begin{align*}
	l \in \set{\text{genomic loci sequenced}} \subset \N
\end{align*}
and let 
\begin{align*}
e : \L &\to [0, 1] \\
l &\mapsto e_l = \P\brac{\text{read is error} \wh \text{position is $l$}}. 
\end{align*}
We assume that the probability of an error read having allele $a \in \A$ to be uniform across $\A$, i.e. 
\begin{align*}
\P_l(a \wh error ) = \frac{1}{4}
\end{align*}



\section{Somatic variant VAF estimate}
Somatic variant calling is performed at per locus level. Fix a locus $l \in \L$. We define
\begin{align*}
 & \hat{a} \in \A \tag*{ (the reference allele at $l$) } \\
 & n = (n_a)_{a \in \A} = (n_A, n_T, n_G, n_C) \tag*{(the count of each allele, or pileup, at $l$)} \\
 & v = (v_a)_{a \in \A} = (v_A, v_T, v_G, v_C) \tag*{(the allele frequency at $l$)} \\
 & V = \sum_{a \in \A} v_a  \tag*{(the sum of the (non-error) allele frequency)}.
\end{align*}
Now, the probability of observing allele $a \in \A$ at position $l$ is given by 
\begin{align}
\P_l(a) 
&= \P_l(a \wh error) \P_l(error) + \P_l(a \cap \overline{error}) \\
&= \frac{1}{4} e_l + v_a  \label{eq: prob allele}
%&= \begin{cases}
%1 - \frac{3}{4} e_l - V &, a = \hat{a} \\
%\frac{1}{4} e_l + v_a &, a \neq \hat{a} 
%\end{cases} 
\end{align}
which give the log-likelihood function in terms of the non-reference allele frequency given the empirical allele counts $n = (n_a)_{a \in \A} $ as
\begin{align}
L(v \wh n) =  \sum_{a \in \A} n_a \log\brac{\frac{1}{4}e_l + v_a} \label{eq: loglikelihood}
%L(v \wh n) = n_{\hat{a}} \log\brac{1 - \frac{3}{4}e_l - V} + \sum_{a \in \A \setminus \set{\hat{a}}} n_a \log\brac{\frac{1}{4}e_l + v_a} \label{eq: loglikelihood}
\end{align}

\subsection{Maximum Likelihood using KKT optimisation}
For maximum likelihood prediction, we want to find 
\begin{align*}
v^* = \argmax_{v \in [0, 1]^3, V \leq 1} L(v \wh n).
\end{align*}
In other words, we have the optimisation problem: maximise  $L(v \wh n)$, subjected to 
\begin{align*}
& - v_a \leq 0 &\quad  \forall a \in \A\\
& V + e_l -1 = v_A, v_T, v_G, v_C + e_l - 1 = 0 &
\end{align*}
the last of which is simply the constraint that (\ref{eq: prob allele}) is a probability measure. Note that these are enough to ensure $v_a \leq 1$ for all $a \in \A$.  
This give rise to the Lagrangian
\begin{align*}
\L(v; \lambda, \mu ) = L(v \wh n) - \mu(V + e_l -1) + \sum_{a \in \A} \lambda_a v_a 
\end{align*}

The corresponding KKT conditions are: 
\begin{align*}
\frac{n_a}{\frac{1}{4}e_l + v_a} &= \mu - \lambda_a \tag*{stationary, $\p_{v_a} \L(v; \lambda, \mu) = 0$}\\
\lambda_a v_a &= 0 \quad \quad \forall a \in \A \tag*{complimentary slackness}\\
\lambda_a &\geq 0 \quad \quad \forall a \in \A \tag*{dual feasibility } \\
V + e_l -1 &= 0 \tag*{primal feasibility.} \\  
\end{align*}
Stationary condition and primal feasibility jointly implies that 
\begin{align}
&v_a = \frac{n_a}{\mu - \lambda_a} - \frac{e_l}{4} \label{cond: 1}\\
&\sum_{a \in \A} \frac{n_a}{\mu - \lambda_a} = 1 \label{cond: 2}
\end{align}
In the case where $v$ is in the interior of the domain, $v \in (0, 1)^4$, we have for all $a \in \A$, $v_a > 0 \implies \lambda_a = 0$, which gives,
\begin{align*}
v_a = \frac{n_a}{n} - \frac{e_l}{4}
\end{align*}
where $n = \sum_{a \in \A} n_a$ is the total allele count at position $l$. In particular, if $e_l = 0$, we have the solution to the uncorrected multinomial maximum likelihood predictor $p_a = n_a / n$. 

We now turn to the case where $v$ lies in one of the lower dimensional boundaries (faces, edges and corners). Let $k$ be the number of $a \in \A$ such that $v_a = 0$. Observe that $\mu - \lambda_a = \frac{4n_a}{e_l}$ whenever, $v_a = 0$ and $\lambda_a = 0$ whenever $v_a \neq 0$ as dictated by the complementary slackness condition. Thus, (\ref{cond: 2}), gives
\begin{align*}
&\sum_{a \in \A} \frac{n_a}{\mu - \lambda_a} = \sum_{v_a = 0} \frac{e_l}{4} + \sum_{v_a \neq 0} \frac{n_a}{\mu}  = 1 \\
\implies& \mu = \frac{ \sum_{v_a \neq 0} n_a}{1 - \frac{ke_l}{4}}.
\end{align*}




%If we further assume that 
%\begin{align*}
%n_a & \geq 0 \quad  \forall a \in \A \\
%\sum_{a \in \A} n_a &> 0 \\
%e_l &\geq 0
%\end{align*}
%
%we can derive the following results. 
%\begin{itemize}
%	\item $\forall a \in \A$, $\lambda_a = 0$. 
%	\begin{proof}
%		If for some $a \in \A$, $\lambda_a > 0$, then $v_a = 0$, and thus, $\lambda_a = -4n_a / e_l \leq 0$ which is a contradiction. 
%	\end{proof}
%	
%	\item $\mu \neq 0$. 
%	\begin{proof}
%		If $\mu = 0$, then $n_a = 0$ for all $a \in \A$ since $\lambda_a = 0$. 
%	\end{proof}
%	\item  $v_a = 0 \iff n_a = 0$. 
%	\begin{proof}
%		If $v_a = 0$, then $\lambda_a = - 4n_a / e_l$. Hence, the dual feasibility condition $\lambda_a \geq 0$ is satisfied if and only if $n_a = 0$. \\
%		Conversely, when $n_a = 0$, we have both $\mu v_a = \lambda_a$ and $\lambda_a v_a = 0$ where the only solution is $\lambda_a = v_a = 0$ since $\mu \neq 0$. \\
%		
%		Note that this result can be derived by plugging $\lambda_a = 0$ from the previous result as well. 
%	\end{proof}
%\end{itemize}

%\subsection{Formula}
%
%If $e_l = 0$, we have 
%\begin{align*}
%n_a = \mu v_a^2 \implies v_a = \sqrt{\frac{n_a}{\mu}}
%\end{align*}
%and the primal feasibility condition gives 
%\begin{align*}
%\frac{1}{\sqrt{\mu}} \sum_{a \in \A} \sqrt{n_a} = 1 \implies \mu = \brac{\sum_{a \in \A} \sqrt{n_a}}^2.
%\end{align*}
%Combining the above, we have, 
%\begin{align*}
%v_a = \frac{\sqrt{n_a}}{\sum_{a \in \A} \sqrt{n_a}}.
%\end{align*}
%
%If $e_l > 0$, 
%\begin{align*}
%\mu v_a^2 + \frac{1}{4}\mu e_l v_a - n_a = 0 \implies v_a = - \frac{1}{8} e_l \brac{1 - \sqrt{1 + \brac{\frac{8}{e_l}}^2 \frac{n_a}{\mu}}}.
%\end{align*}
%And primal feasibility in this case gives
%\begin{align*}
%\sum_{a \in \A} \sqrt{1 + \brac{\frac{8}{e_l}}^2 \frac{n_a}{\mu}} = 4 \brac{\frac{2}{e_l} - 1}. 
%\end{align*}
%For terms with $n_a > 0$, if we assume $\mu \ll \brac{\frac{8}{e_l}}^2 n_a$, then we can approximate the term by the first term in the Taylor expansion of $\sqrt{1 + x}$ at $x = \infty$, giving 
%\begin{align*}
%\frac{8}{e_l} \frac{\sum_{a \in \A} \sqrt{n_a}}{\sqrt{\mu}} = 4\brac{\frac{2}{e_l} -1} -k \implies \mu = \brac{\frac{\sum_{a \in \A} \sqrt{n_a}}{1 - \frac{4 + k}{8} e_l }}^2
%\end{align*}
%where $k$ is the number of $a \in \A$ with $n_a = 0$. This then give 
%\begin{align*}
%v_a 
%= - \frac{1}{8} e_l \brac{1 - \sqrt{1 + \brac{\frac{ 4 \brac{\frac{2}{e_l} - 1} - k}{\sum_{a \in \A} \sqrt{n_a}}}^2 n_a}} 
%=  - \frac{1}{8} e_l \brac{1 - \sqrt{1 + \sqbrac{4 \brac{\frac{2}{e_l} - 1} - k}^2 \mathring{v}_a^2}}
%\end{align*}
%where $\mathring{v}_a$ is the solution for when $e_l = 0$. 

\subsection{Likelihood ratio test}
Under the null-hypothesis that there is no variant at position $l$, the probability of observing allele $ a \in \A$ is given by 
\begin{align*}
\P_l(a) = \begin{cases}
1 - \frac{3}{4} e_l & , a = \hat{a} \\
\frac{1}{4} e_l & , a \neq \hat{a}
\end{cases}
\end{align*}
where $\hat{a} \in \A$ is the reference allele at position $l$. Thus, the log-likelihood function is given by 
\begin{align*}
L_0(n) = n_{\hat{a}} \log\brac{1 - \frac{3}{4} e_l } + \log\brac{\frac{1}{4} e_l } \sum_{a \in \A \setminus{\hat{a}}} n_a. 
\end{align*}
The likelihood ratio test statistics is then given by 
\begin{align*}
G = 2 \times \brac{L(v^* \wh n) - L_0(n)}
\end{align*}
and its distribution is given by $\chi^2$ distribution with 3 degrees of freedom\footnote{The degree of freedom for the null model is zero while that of the alternative is 3, given by 4 parameters which sum to a given constant.}.

\end{document}